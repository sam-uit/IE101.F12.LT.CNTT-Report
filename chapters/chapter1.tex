\chapter{GIỚI THIỆU ĐỀ TÀI}
\label{cha:chapter01}

Trong bối cảnh Cách Mạng Công Nghiệp 4.0 và chuyển đổi số của quốc gia đến năm 2026. Cơ Sở Hạ Tầng Công Nghệ Nhông Tin (IT Infrastucture) là một trong những yếu tố cốt lõi quyết định nên sự phát triển và khả năng cạnh tranh của các doanh nghiệp, đặc biệt là các doanh nghiệp phát triển phần mềm.

Theo chương trình môn học ``Cơ Sở Tạ Tầng Công Nghệ Thông Tin'' tại Trường Đại Học Công Nghệ Thông Tin, hạ tầng công nghệ thông tin không chỉ bao gồm các thành phần vật lý mà còn bao gồm các phần mềm, quản trị dữ liệu và an ninh mạng. Tất cả các thành phần trên đều phải được thiết kế để đảm bảo tính sẵn sàng, khả năng mở rộng, bảo mật và hiệu quả chi phí.

Việc nghiên cứu cơ sở hạ tầng công nghệ thông tin của một doanh nghiệp thực tế như Công ty phần mềm \textbf{Orient Software Development Corporation} giúp nhóm áp dụng được các khái niệm lý thuyết vào phân tích thực tiễn, từ đó có thể hiễu rõ hơn về cách một doanh nghiệp lớn triển khai và quản lý hệ thống hạ tầng công nghệ thông tin như thế nào.

\section{Lý do chọn đề tài} % (fold)
\label{sec:de-tai}

Lý do chọn đề tài ``Phân tích Cơ sở Hạ tầng Công nghệ Thông tin của Công ty phần mềm Orient Software Development Corporation'' bao gồm:

\begin{newitemize}
    \item \textbf{Tầm quan trong của hạ tầng công nghệ thông tin:} hạ tầng công nghệ thông tin là nền tảng cơ bản cho quá trình chuyển đổi số, tự động hóa các quy trình, nghiệp vụ và bảo vệ dữ liệu, cung cấp tài nguyên cho tất cả các bộ phận, nhân viên trong quá trình làm việc.
    \item \textbf{Tính thực tiễn:} Công ty phần mềm Orient Sofatware Development Corp là một doanh nghiệp lớn về mảng thiết kế và phát triển phần mềm, đa số đều là các khách hàng nước ngoài và yêu cầu quá trình bảo mật dữ liệu cực kỳ cao. Qua đó cho thấy việc cơ sở hạ tầng công nghệ thông tin là “xương sống” trong mọi hoat động phát triển, kiểm thử và triển khai phần mềm. Hiểu rõ và phân tích hạ tầng này giúp nhóm nghiên cứu nắm bắt cách một doanh nghiệp đa dự án, đa công nghệ vận hành hiệu quả trong môi trường cạnh tranh toàn cầu.
    \item \textbf{Cơ hội học tập và phát triển:}
        \begin{newitemize}
            \item [\textbullet] \textbf{Sự phức tạp và đa dạng về công nghệ:} hạ tầng của một công ty outsourcing buộc phải linh hoạt để hỗ trợ nhiều dự án với các yêu cầu kỹ thuật khác nhau (đa nền tảng, đa ngôn ngữ lập trình, đa môi trường Cloud như AWS, Azure, GCP). Qua đó đề tài này giúp nhóm nghiên cứu sâu về các công nghệ hiện đại mà doanh nghiệp lớn đang sử dụng trong qua trình phát triển.
            \item [\textbullet] \textbf{Vấn đề bảo mật và tối ưu chi phí:} trong ngành phát triển phần mềm, việc đảm bảo an toàn và bảo mật dữ liệu khách hàng là ưu tiên hàng đầu. Đồ án cho phép phân tích các giải pháp bảo mật mạng, kiểm soát truy cập và phòng chống thảm họa (Disaster Recovery). Ngoài ra, hạ tầng công nghệ thông tin là một khoảng đầu tư rất lớn trong một doanh nghiệp, việc phân tích và đưa ra các đề xuất về tối ưu hóa tài nguyên và quản lý chi phí sẽ mạng lại trị kinh tế và ứng dụng cao cho công ty.
        \end{newitemize}
\end{newitemize}

\section{Mục tiêu nghiên cứu} % (fold)
\label{sec:muc-tieu}

Đồ án này đặt ra các mục tiêu nghiên cứu cơ bản sau:

\begin{newitemize}
    \item \textbf{Phân tích hiện trạng:} mô tả và phân tích chi tiết kiến trúc tổng thể, các thành phần chính (mạng, máy chủ, lưu trữ, môi trường Cloud) và các quy trình vận hành cơ sở hạ tầng công nghệ thông tin hiện tại của Công ty phần mềm Orient Software Development Corporation.
    \item \textbf{Đánh giá:} Đánh giá ưu điểm, nhược điểm và các vấn đề còn tồn tại của hệ thống hạ tầng hiện hữu, đặc biệt về các mặt: hiệu suất, độ tin cậy (High Availability), khả năng mở rộng (Scalability) và bảo mật.
    \item \textbf{Đề xuất giải pháp:} đưa ra các đề xuất cải tiến và tối ưu hóa hạ tầng công nghệ thông tin nhằm nâng cao hiệu quả hoạt động, tăng cường bảo mật và tối ưu chi phí, đáp ứng tốt hơn cho nhu cầu phát triển và gia công phần mềm trong tương lai.
\end{newitemize}

\section{Phạm vi nghiên cứu} % (fold)
\label{sec:pham-vi-nghien-cuu}

Để đảm bảo tính khả thi và tập trung của đồ án, phạm vi nghiên cứu được xác định như sau:

\begin{newitemize}
    \item \textbf{Đối tượng nghiên cứu:} cơ sở hạ tầng công nghệ thông tin phục vụ hoạt động phát triển và triển khai phần mềm (Development \& Delivery Infrastructure) tại Công ty phần mềm Orient Software Development Corporation. Phần cứng, phần mềm, tài nguyên mạng, an ninh, bảo mật và quản trị dữ liệu.
    \item \textbf{Giới hạn về nội dung:} Đồ án tập trung vào các khía cạnh chính của hạ tầng:

    \begin{newitemize}
        \item [\textbullet] Kiến trúc mạng nội bộ và mạng kết nối Internet.
        \item [\textbullet] Các hệ thống máy chủ và giải pháp ảo hóa.
        \item [\textbullet] Các phần mềm quản lý người dùng đầu cuối và bảo mật hệ thống máy tính.
        \item [\textbullet] Các dịch vụ Cloud được sử dụng để quản lý dự án, lưu trữ code và triển khai sản phẩm.
        \item [\textbullet] Các chính sách và giải pháp bảo mật hạ tầng cốt lõi.
    \end{newitemize}

    \item \textbf{Giới hạn về không gian và thời gian:} Nghiên cứu được thực hiện tại trụ sở chính hoặc chi nhánh lớn nhất của Công ty phần mềm Orient Software Development Corporation tại Việt Nam (TP. Hồ Chí Minh).
\end{newitemize}

Thời gian thu thập dữ liệu và phân tích được giới hạn trong khoảng từ 01/11/2025 đến 01/12/2025 của giai đoạn thực hiện đồ án.

\section{Giới thiệu sơ lược về Orient Software Development Corp} % (fold)
\label{sec:gioi-thieu-orient-software}

\begin{table}[htbp]
    \centering
    \begin{tabular}{p{0.2\textwidth} | p{0.7\textwidth}}
       \toprule
        \textbf{Tên Công ty} & Orient Software Development Corporation \\
                \midrule
        Website & \texttt{\href{https://www.orientsoftware.com/}{www.orientsoftware.com}} \\
        \midrule
        Năm thành lập & 2005 (thành lập tại Singapore, sau đó phát triển mạnh mẽ tại Việt Nam) \\
        \midrule
        Lĩnh vực hoạt động & Gia công phần mềm (Software Outsourcing),\newline Phát triển phần mềm theo yêu cầu,\newline Tư vấn công nghệ, và gần đây tập trung vào AI \& Data. \\
        \midrule
        Tầm vóc & Được đánh giá là một trong những công ty outsourcing hàng đầu tại Việt Nam. \\
        \midrule
        Quy mô nhân sự & Hiện có hơn 350+ chuyên gia kỹ thuật (với mục tiêu đạt 1000 người trong thời gian tới). \\
        \midrule
        Các địa điểm chính & Có văn phòng tại TP. Hồ Chí Minh (Thủ Đức, Tân Bình), Đà Nẵng, và Hà Nội, cùng các văn phòng đại diện tại nước ngoài (Mỹ, Úc, Nhật Bản). \\
        \midrule
        Sản phẩm/Dịch vụ tiêu biểu & Cung cấp dịch vụ phát triển phần mềm toàn diện (End-to-end Software Services) cho khách hàng từ Startup, Doanh nghiệp vừa và nhỏ (SMB) đến các Tập đoàn lớn (Enterprise) trên toàn thế giới. \\
        \midrule
        Sứ mệnh & Trở thành đối tác kỹ thuật đáng tin cậy, cung cấp các giải pháp phần mềm chuyên nghiệp, đổi mới, và chất lượng cao, đặt con người và sự hợp tác làm trọng tâm. \\
        \bottomrule
    \end{tabular}
    \caption{Orient Software Development Corp}
    \label{tab:orientsoftware}
\end{table}

