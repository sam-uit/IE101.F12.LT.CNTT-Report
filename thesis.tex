%%===========================================================================%%
%% thesis.tex -- MAIN FILE (the one that you compile with LuaLaTeX)
%%===========================================================================%%

% Set up the document for LuaLaTeX with native Unicode support
\documentclass[a4paper, oneside, 13pt]{thesis}
% Modern packages - UTF-8 and Vietnamese are natively supported in LuaLaTeX
\usepackage{blindtext}
\usepackage[all]{hypcap}
\usepackage{paralist}
\usepackage{eqparbox}
\usepackage{array}
\usepackage{multirow}
\usepackage{booktabs}
\usepackage{tabularx}
\usepackage{float}
\usepackage{calc}
\usepackage{tocloft}

%\usemintedstyle{trac}
\floatstyle{boxed}
\restylefloat{figure}

% Display the chapter name (i.e Chương) in TOC
\renewcommand\cftchappresnum{\chaptername\space}
\renewcommand\cftchapaftersnum{.\space}

% Set a width for chapnum that can span wider after we add the \chaptername
\setlength{\cftchapnumwidth}{\widthof{\textbf{Chương~999~}}}

\makeatletter
\g@addto@macro\appendix{%
  \addtocontents{toc}{%
    \protect\renewcommand{\protect\cftchappresnum}{\appendixname\space}%
  }%
}
\makeatother
% %   % execute this command before entering appendix area
%     \setcounter{chapter}{0}
%     \renewcommand\thechapter{\appendixname~\Alph\space{chapter}}}
%%===========================================================================%%
\makeglossaries{}
%==============================================================================
\graphicspath{{figures/}}  % Location of the graphics files
%%===========================================================================%%
%\usepackage{etoolbox}
%\makeatletter
%\preto{\@newitemize}{\topsep=0pt \partopsep=0pt }
%\makeatother
% Include any extra LaTeX packages required
\usepackage[square, numbers, comma, sort&compress]{natbib}  % Use the "Natbib" style for the references in the Bibliography
\usepackage{verbatim}  % Needed for the "comment" environment to make LaTeX comments
\usepackage{vector}  % Allows "\bvec{}" and "\buvec{}" for "blackboard" style bold vectors in maths
%\hypersetup{urlcolor=blue, colorlinks=true}  % Colours hyperlinks in blue, but
%this can be distracting if there are many links.
%==============================================================================

%==============================================================================

\hypersetup{
%     %bookmarks=true,         % show bookmarks bar?
%     %unicode=false,          % non-Latin characters in Acrobat’s bookmarks
%     %pdftoolbar=true,        % show Acrobat’s toolbar?
%     %pdfmenubar=true,        % show Acrobat’s menu?
%     %pdffitwindow=false,     % window fit to page when opened
%     %pdfstartview={FitH},    % fits the width of the page to the window
     pdftitle={THE DOCUMENT TITLE},    % title
     pdfauthor={author},     % author
     pdfsubject={subject},   % subject of the document
     pdfcreator={creator},   % creator of the document
     pdfproducer={producer}, % producer of the document
     pdfkeywords={keywords}, % list of keywords
%     %pdfnewwindow=true,      % links in new window
     colorlinks=true,       % false: boxed links; true: colored links
%     linkcolor=blue,          % color of internal links
%     citecolor=green,        % color of links to bibliography
%     filecolor=magenta,      % color of file links
%     urlcolor=cyan           % color of external links
}

%==============================================================================
\newenvironment{newenumerate}
    {\begin{enumerate}
	\setlength{\itemsep}{1pt}
	\setlength{\parskip}{0pt}
	\setlength{\parsep}{0pt}
	\setlength{\topsep}{0pt}
	\setlength{\partopsep}{0pt}}
    {\end{enumerate}}

\newenvironment{newitemize}
    {\begin{itemize}
	\setlength{\itemsep}{1pt}
	\setlength{\parskip}{0pt}
	\setlength{\parsep}{0pt}
	\setlength{\topsep}{0pt}
	\setlength{\partopsep}{0pt}}
    {\end{itemize}}

\newenvironment{newdescription}
    {\begin{description}
	\setlength{\itemsep}{1pt}
	\setlength{\parskip}{0pt}
	\setlength{\parsep}{0pt}
	\setlength{\topsep}{0pt}
	\setlength{\partopsep}{0pt}}
    {\end{description}}

%==============================================================================
\begin{document} % BEGIN
%==============================================================================

\frontmatter	  % Begin Roman style (i, ii, iii, iv...) page numbering

% Set up the Title Page
\title{\bf\Huge BÁO CÁO MÔN HỌC}
\authors{Nhóm 1 (Alpha)}
%\addresses  {\groupname\\\deptname\\\univname}  % Do not change this here
\addresses{TP.~Hồ Chí Minh}
\date{Tháng 12, 2025}
\subject{PHÂN TÍCH CƠ SỞ HẠ TẦNG CÔNG NGHỆ THÔNG TIN CỦA MỘT DOANH NGHIỆP}
\keywords{IT103}
\instructor{Ths. Đỗ Minh Tiến}
\INSTRUCTOR{Ths. Đỗ Minh Tiến}
\student{Nhóm 1 (Alpha)}
\STUDENT{Nhóm 1 (Alpha)}
\projectname{PHÂN TÍCH CƠ SỞ HẠ TẦNG CÔNG NGHỆ THÔNG TIN CỦA MỘT DOANH NGHIỆP}
\department{Cơ Sở Hạ Tầng CNTT}
\DEPARTMENT{IT103}
\faculty{FACULTY}
\FACULTY{FACULTY}
\class{LT.K2025.2-CNTT}
\schoolyear{2025}

%==============================================================================
\maketitle % MAKE TITLE
%==============================================================================

\setstretch{1.3}  % It is better to have smaller font and larger line spacing than the other way round

% Define the page headers using the FancyHdr package and set up for one-sided printing
\fancyhead{}  % Clears all page headers and footers
\rhead{\thepage}  % Sets the right side header to show the page number
% Hiện tiêu đề Mục Lục trên header
\lhead{\textsc{\leftmark}}

\pagestyle{fancy}  % Finally, use the "fancy" page style to implement the FancyHdr headers

%==============================================================================
% An intentionally blank page
%\newpage
\pagestyle{empty}
\null\vfill
\begin{center}
    Trang này được cố tình để trống.
\end{center}
\clearpage
\pagestyle{fancy}

%==============================================================================
% Declaration Page required for the Thesis, your institution may give you a different text to place here
\Declaration{
    \addtocontents{toc}{}  % Add a gap in the Contents, for aesthetics
    \begin{table}[htbp]
    \centering
    \begin{tabular}{p{0.1\textwidth} | p{0.2\textwidth} | p{0.3\textwidth}| p{0.3\textwidth}}
       \toprule
        \textbf{MSSV} & \textbf{Họ Tên} & \textbf{Nhiệm Vụ} & \textbf{Chương Mục} \\
        \midrule
        \midrule
        25410171 & Lê Thị Tú Anh & Thiết kế slide Làm bài tập chương  & Chương I, II, III, IV, V \\
        25410179 & Giang Hải Chương & Làm bài tập chương  Phản biện & Chương III \\
        25410183 & Nguyễn Đào Anh Đạt & Thuyết trình & Chương V \\
        25410193 & Nguyễn Minh Duy & Thiết kế slide & Chương I, II, III, IV, V \\
        25410204 & Trương Xuân Hậu & Phản biện & Chương V \\
        25410213 & Phan Chí Hiếu & Làm bài tập chương Phản biện & Chương IV \\
        25410220 & Vũ Huy Hoàng & Làm bài tập chương  Thuyết trình & Chương IV \\
        25410239 & Phạm Tuấn Kiệt & Làm bài tập chương  Phản biện & Chương V \\
        25410240 & Nguyễn Tạ Quí Lan & Thiết kế slide  Chỉnh file đồ án Phản biện & Chương I, II, III, IV, V Chương III \\
        25410244 & Nguyễn Thành Lộc & Làm bài tập chương  Phản biện & Chương V \\
        25410291 & Đinh Xuân Sâm & Chỉnh file đồ án Thuyết trình & Chương I, II \\
        25410319 & Đặng Hữu Toàn & Làm bài tập chương  Phản biện & Chương IV \\
        25410321 & Nguyễn Điền Triết & Làm bài tập chương  Thuyết trình & Chương III \\
        25410325 & Nguyễn Văn Trung & Tìm thông tin và nội dung công ty Làm file đồ án Phản biện & Chương I, II, III, IV, V Chương I, II \\
        25410338 & Lê Anh Vũ & Phản biện & Chương V \\
   \bottomrule
    \end{tabular}
    \caption{Nhóm 1 (Alpha) -- Bảng Phân Công}
    \label{tab:alphagroup}
\end{table}

}
\clearpage  % Declaration ended, now start a new page
%==============================================================================

% An empty page: 1st
\pagestyle{empty}  % No headers or footers for the following pages
\null\vfill
\begin{center}
    Trang này được cố tình để trống.
\end{center}
\clearpage  % Funny Quote page ended, start a new page
%==============================================================================

\setstretch{1.3}  % Reset the line-spacing to 1.3 for body text (if it has
%changed)

% The Acknowledgements page, for thanking everyone
\acknowledgements{
    \addtocontents{toc}{}  % Add a gap in the Contents, for aesthetics
    Chúng em xin gửi lời cảm ơn đến thầy Đỗ Minh Tiến đã tận tình hướng dẫn, cung cấp kiến thức chuyên sâu và đưa ra những góp ý quý báu để nhóm chúng em hoàn thiện đề tài một cách chỉnh chu và chuyên nghiệp. Sự hỗ trợ của thầy không chỉ giúp chúng em áp dụng kiến thức lý thuyết vào thực tiễn mà còn truyền cảm hứng để nhóm chúng em khám phá sâu hơn về lĩnh vực hạ tầng Công Nghệ Thông Tin.

Chúng em xin gửi lời cảm ơn đến Trường Đại học Công nghệ Thông tin -- Đại học Quốc gia TP. Hồ Chí Minh và Trung tâm Phát triển Công nghệ Thông tin đã tạo điều kiện thuận lợi về tài liệu, cơ sở vật chất và môi trường học tập để nhóm chúng em có thể thực hiện nghiên cứu.

Chúng em xin gửi lời cảm ơn chân thành đến Công ty Orient Software Development Coporation đã tạo điều kiện để chúng em có thể tìm hiểu về công ty. Thông qua các tài liệu chính thức, báo cáo thường niên và thông tin công khai trên website đã cung cấp nguồn dữ liệu phong phú và đáng tin cậy. 

Cuối cùng, xin cảm ơn các thành viên đã cùng nhau phối hợp có trách nhiệm để hoàn thiện đồ án đúng tiến độ và đạt chất lượng tốt nhất. Mặc dù đã cố gắng hết sức, nhưng không tránh khỏi những thiếu sót chúng em mong nhận được sự góp ý từ Thầy để nhóm chúng em có thể rút kinh nghiệm và cải thiện trong các dự án sau.

Xin chân thành cảm ơn!

}
\clearpage  % End of the Acknowledgements
%==============================================================================

% An empty page: 2nd
\pagestyle{empty}  % No headers or footers for the following pages
\null\vfill
\begin{center}
    Trang này được cố tình để trống.
\end{center}
\clearpage  % Funny Quote page ended, start a new page
%==============================================================================

\setstretch{1.3}

\reviews{
    \addtocontents{toc}{}
}

\null\vfill

Đã ký:\\
\rule[1em]{25em}{0.5pt}

Ngày:\\ \rule[1em]{25em}{0.5pt}

\clearpage
%==============================================================================

% An empty page: 3rd
\pagestyle{empty}
\null\vfill
\begin{center}
    Trang này được cố tình để trống.
\end{center}
\clearpage
%==============================================================================

\pagestyle{fancy}

%------------------------------------------------------------------------------
\addtotoc{Mục Lục}
\tableofcontents
\clearpage
% TODO: Display the page number (roman style) in all the pages of table of contents, not only the very first page.
%------------------------------------------------------------------------------
\addtotoc{Danh Sách Hình Ảnh}
\renewcommand*\listfigurename{Danh Sách Hình Ảnh}
\listoffigures
\clearpage
%------------------------------------------------------------------------------
\addtotoc{Danh Sách Bảng}
\listoftables
\clearpage
%------------------------------------------------------------------------------
\addtotoc{Các Đoạn Mã Nguồn}
\renewcommand\listoflistingscaption{Các Đoạn Mã Nguồn}
\listoflistings{}
%------------------------------------------------------------------------------
\newglossaryentry{upload}
{
    name=Upload,
    description={Thao tác, quá trình tải dữ liệu lên máy chủ Web}
}

\newglossaryentry{www}
{
    name=WWW,
    description={World Wide Web}
}

\newglossaryentry{irc}
{
    name=IRC,
    description={\textbf{I}nternet \textbf{R}elay \textbf{C}hat}
}

\newglossaryentry{hdd}
{
    name=HDD,
    description={Hard Disk Driver}
}

\newglossaryentry{ubuntu}
{
    name=Ubuntu,
    description={Một bản phân phối \textbf{Linux} phổ biến}
}

\newglossaryentry{exif}
{
    name={Exif},
    description={\textbf{Ex}changeable \textbf{I}mage \textbf{F}ile Format}
}

\newglossaryentry{ide}
{
    name={IDE},
    description={Integrated Development Enviroment - Môi trường phát triển tích
    hợp}
}

\newglossaryentry{rest}
{
    name={REST},
    description={\textbf{Re}presentational \textbf{S}tate \textbf{T}ransfer -
    Một kiểu kiến trúc phần mềm dành cho các hệ thống phân tán như WWW}
}

\newglossaryentry{restful}
{
    name={RESTful},
    description={RESTful web API - Một kiểu xây dựng dịch vụ Web}
}

\newglossaryentry{http}
{
    name={HTTP},
    description={\textbf{H}ypertext \textbf{T}ransfer \textbf{P}rotocol}
}

\newglossaryentry{https}
{
    name={HTTPS},
    description={\textbf{H}ypertext \textbf{T}ransfer \textbf{P}rotocol
        \textbf{S}ecure (\textbf{SSL/TLS})}
}

\newglossaryentry{xml}
{
    name={XML},
    description={e\textbf{X}tensible \textbf{M}arkup \textbf{L}anguage ---
    Ngôn ngữ đánh dấu mở rộng}
}

\newglossaryentry{json}
{
    name={JSON},
    description={\textbf{J}ava\textbf{S}cript \textbf{O}bject \textbf{N}otation}
}

\newglossaryentry{oauth}
{
    name={OAuth},
    description={Một chuẩn mở cho việc xác thực và ủy quyền khả năng truy cập
    tài nguyên riêng tư}
}

\newglossaryentry{gui}
{
    name={GUI},
    description={\textbf{G}raphical \textbf{U}ser \textbf{I}nterface -- Giao diện
    người dùng đồ họa}
}

\newglossaryentry{agile}
{
    name={Agile},
    description={Phương pháp phát triển phần mềm linh hoạt}
}

\newglossaryentry{waterfall}
{
    name={Waterfall},
    description={Một phương pháp phát triển phần mềm được ứng dụng phổ biến}
}

\newglossaryentry{oop}
{
    name={OOP},
    description={\textbf{O}bject-\textbf{O}riented \textbf{P}rogramming}
}

\newglossaryentry{rad}
{
    name={RAD},
    description={\textbf{R}apid \textbf{A}pplication \textbf{D}evelopment}
}

\newglossaryentry{url}
{
    name={URL},
    description={\textbf{U}niform \textbf{R}esource \textbf{L}ocator}
}

\newglossaryentry{api}
{
    name={API},
    description={\textbf{A}pplication \textbf{P}rogramming \textbf{I}nterface}
}

\newglossaryentry{scm}
{
    name={SCM},
    description={\textbf{S}ource \textbf{C}ode \textbf{M}anager}
}

\newglossaryentry{drcs}
{
    name={DRCS},
    description={\textbf{D}istrubuted \textbf{R}evision
    \textbf{C}ontrol \textbf{S}ystem}
}

\newglossaryentry{foss}
{
    name={FOSS},
    description={\textbf{F}ree \textbf{O}pen \textbf{S}ource \textbf{S}otfware
    -- \textbf{FREE} \emph{as} \textbf{FREEDOM}}
}

\setglossarystyle{altlist}
\printglossarieshere{}
\clearpage
%==============================================================================

\setstretch{1.3}  % Return the line spacing back to 1.3

\pagestyle{empty}  % Page style needs to be empty for this page
\dedicatory{}
% \addtocontents{toc}{\vspace{2em}}  % Add a gap in the Contents, for aesthetics

%==============================================================================
\mainmatter{} % Begin normal, numeric (1,2,3...) page numbering
\pagestyle{fancy}
\renewcommand{\chaptermark}[1]{\markboth{#1}{}}
\renewcommand{\sectionmark}[1]{\markright{#1}{}}
\lhead{\textsc{\chaptername} \space \thechapter. \space \textsc{\leftmark}}
\renewcommand{\footrulewidth}{0.4pt} % default is 0pt
\lfoot{\small GVHD:~\instructorname}\rfoot{\small SVTH:~\studentname}
% Include the chapters of the thesis, as separate files
% Just uncomment the lines as you write the chapters

%==============================================================================
% \chapter{Giới thiệu}
\label{chap:intro}
Lorem ipsum dolor sit amet, consectetur adipiscing elit. Sed do eiusmod tempor incididunt ut labore et dolore magna aliqua.

\section{Mục tiêu}
Ut enim ad minim veniam, quis nostrud exercitation ullamco laboris nisi ut aliquip ex ea commodo consequat.

\section{Cấu trúc đồ án}
Duis aute irure dolor in reprehenderit in voluptate velit esse cillum dolore eu fugiat nulla pariatur.
 % Introduction
\chapter{Giới thiệu đề tài}
\label{cha:chapter01}

In \emph{Dịch vụ mạng RESTful: Khái niệm cơ bản} (\citep{ibmrestfulbasic}), we can have a brief on \gls{restful}.

\blindtext

\section{Đề tài} % (fold)
\label{sec:de-tai}

\blindtext

\section{Mục đích} % (fold)
\label{sec:muc-dich}

\blindtext

\section{Kết quả} % (fold)
\label{sec:ket-qua}

\blindtext

\begin{newitemize}
    \item Xây dựng được ứng dụng với những chức năng cơ bản.
    \item Tìm hiểu và nắm bắt những kiến thức thực tế, kỹ năng cần có trong suốt quá trình thực hiện đề tài.
\end{newitemize}

\section{Bố cục} % (fold)
\label{sec:bo-cuc}

\blindtext

\begin{center}
    \line(1,0){250}
\end{center}

\blindtext

\chapter{THÀNH PHẦN CẤU TẠO CƠ SỞ HẠ TẦNG CNTT}
\label{cha:chapter2}

\section{Hạ tầng phần cứng}
\label{sec:section2.1}

\subsection{Thiết bị mạng}
\label{subsec:subsection2.1.1}

\begin{table}[H]
    \centering
    \begin{tabular}{p{0.2\textwidth} | p{0.7\textwidth}}
        \toprule
        \textbf{Thành Phần} & \textbf{Mô Tả \& Mục Đích}                                                                                                                                                                                                                                                         \\
        \hline \hline
        Firewall            & \textbullet \space Bảo vệ mạng nội bộ khỏi các mối đe dọa từ bên ngoài Internet. \newline \textbullet \space Kiểm soát lưu lượng mạng vào/ra dựa trên các quy tắc ưu tiên QoS (Quality of Service). \newline \textbullet \space Ngăn chặn truy cập trái phép, phần mềm độc hại.    \\
        \midrule
        Switch              & \textbullet \space Kết nối các thiết bị trong cùng một mạng nội bộ (LAN) với nhau. \newline \textbullet \space Chuyển tiếp dữ liệu một cách có chủ đích giữa các thiết bị với nhau. \newline \textbullet \space Là thiết bị đóng vài trò tạo nên xương sống của hệ thống mạng LAN. \\
        \midrule
        Router              & \textbullet \space Thiết bị dùng để cung cấp Internet của nhà mạng, dùng để kết nối vào Firewall để các thiết bị có thể truy cập Internet.                                                                                                                                         \\
        \midrule
        Access Point        & \textbullet \space Phát sóng Wi-Fi, cho phép các thiết bị không dây kết nối với nhau qua mạng nội bộ. \newline \textbullet \space Mở rộng vòng phủ sóng của mạng.                                                                                                                  \\
        \bottomrule
    \end{tabular}
    \caption{Phần Cứng -- Thiết Bị Mạng}
    \label{tab:hardwarenetwork}
\end{table}

\subsection{Hệ thống máy chủ (Server) và Lưu Trữ}
\label{subsubsec:subsubsection2.1.1.1}

\begin{table}[H]
    \centering
    \begin{tabular}{p{0.2\textwidth} | p{0.7\textwidth}}
        \toprule
        \textbf{Thành Phần}      & \textbf{Mô Tả \& Mục Đích}                                                                                                                                                                                                                \\
        \hline \hline
        Active Directory Server  & \textbullet \space Quản lý truy cập và phân quyền cho tất cả các thiết bị đầu cuối và người dùng trong hệ thống nội bộ. \newline \textbullet \space Là xương sống trong quản lý hệ thống để liên kết với các dịch vụ khác (Azure, Cloud). \\
        \midrule
        Máy chủ ảo hóa (Hyper-V) & \textbullet \space Dựa trên công nghệ ảo hóa của Microsoft. \newline \textbullet \space Triển khai công cụ ảo hóa các dịch vụ và máy chủ khác để tối ưu hóa chi phí và tài nguyên.                                                        \\
        \midrule
        Lưu Trữ                  & \textbullet \space Sử dụng Synology NAS nội bộ và Cloud Storage (Onedrive). \newline \textbullet \space Lưu trữ và truy cập các dữ liệu nhanh chóng và tiện lợi.                                                                          \\
        \midrule
        DNS Server               & \textbullet \space Sử dụng DNS Server nội bộ để truy cập các dịch vụ dễ dàng hơn. \newline \textbullet \space Ngoài ra còn sử dụng DNS Cloudflare.                                                                                        \\
        \bottomrule
    \end{tabular}
    \caption{Phần Cứng - Máy Chủ \& Lưu Trữ}
    \label{tab:hardwareserver}
\end{table}

\subsection{Thiết bị đầu cuối}
\label{subsec:subsection2.1.2}

\begin{table}[H]
    \centering
    \begin{tabular}{p{0.2\textwidth} | p{0.7\textwidth}}
        \toprule
        \textbf{Thành Phần} & \textbf{Mô Tả \& Mục Đích}                                                                                                                                                                                                                                                         \\
        \hline \hline
        Firewall            & \textbullet \space Bảo vệ mạng nội bộ khỏi các mối đe dọa từ bên ngoài Internet. \newline \textbullet \space Kiểm soát lưu lượng mạng vào/ra dựa trên các quy tắc ưu tiên QoS (Quality of Service). \newline \textbullet \space Ngăn chặn truy cập trái phép, phần mềm độc hại.    \\
        \midrule
        Switch              & \textbullet \space Kết nối các thiết bị trong cùng một mạng nội bộ (LAN) với nhau. \newline \textbullet \space Chuyển tiếp dữ liệu một cách có chủ đích giữa các thiết bị với nhau. \newline \textbullet \space Là thiết bị đóng vài trò tạo nên xương sống của hệ thống mạng LAN. \\
        \midrule
        Router              & \textbullet \space Thiết bị dùng để cung cấp Internet của nhà mạng, dùng để kết nối vào Firewall để các thiết bị có thể truy cập Internet.                                                                                                                                         \\
        \midrule
        Access Point        & \textbullet \space Phát sóng Wi-Fi, cho phép các thiết bị không dây kết nối với nhau qua mạng nội bộ. \newline \textbullet \space Mở rộng vòng phủ sóng của mạng.                                                                                                                  \\
        \bottomrule
    \end{tabular}
    \caption{Phần Cứng - Thiết Bị Đầu Cuối}
    \label{tab:hardwareenddevices}
\end{table}

\section{Hạ tầng phần mềm}

\subsection{Phần mềm nền tảng}

\begin{table}[H]
    \centering
    \begin{tabular}{p{0.2\textwidth} | p{0.7\textwidth}}
        \toprule
        \textbf{Thành Phần} & \textbf{Mô Tả \& Mục Đích}                                                                                                                                                                                                                                                         \\
        \hline \hline
        Firewall            & \textbullet \space Bảo vệ mạng nội bộ khỏi các mối đe dọa từ bên ngoài Internet. \newline \textbullet \space Kiểm soát lưu lượng mạng vào/ra dựa trên các quy tắc ưu tiên QoS (Quality of Service). \newline \textbullet \space Ngăn chặn truy cập trái phép, phần mềm độc hại.    \\
        \midrule
        Switch              & \textbullet \space Kết nối các thiết bị trong cùng một mạng nội bộ (LAN) với nhau. \newline \textbullet \space Chuyển tiếp dữ liệu một cách có chủ đích giữa các thiết bị với nhau. \newline \textbullet \space Là thiết bị đóng vài trò tạo nên xương sống của hệ thống mạng LAN. \\
        \midrule
        Router              & \textbullet \space Thiết bị dùng để cung cấp Internet của nhà mạng, dùng để kết nối vào Firewall để các thiết bị có thể truy cập Internet.                                                                                                                                         \\
        \midrule
        Access Point        & \textbullet \space Phát sóng Wi-Fi, cho phép các thiết bị không dây kết nối với nhau qua mạng nội bộ. \newline \textbullet \space Mở rộng vòng phủ sóng của mạng.                                                                                                                  \\
        \bottomrule
    \end{tabular}
    \caption{Phần Mềm - Nền Tảng}
    \label{tab:softwareplatform}
\end{table}

\subsection{Phần mềm quản lý}

\begin{table}[H]
    \centering
    \begin{tabular}{p{0.2\textwidth} | p{0.7\textwidth}}
        \toprule
        \textbf{Thành Phần} & \textbf{Mô Tả \& Mục Đích}                                                                                                                                                                                                                                                         \\
        \hline \hline
        Firewall            & \textbullet \space Bảo vệ mạng nội bộ khỏi các mối đe dọa từ bên ngoài Internet. \newline \textbullet \space Kiểm soát lưu lượng mạng vào/ra dựa trên các quy tắc ưu tiên QoS (Quality of Service). \newline \textbullet \space Ngăn chặn truy cập trái phép, phần mềm độc hại.    \\
        \midrule
        Switch              & \textbullet \space Kết nối các thiết bị trong cùng một mạng nội bộ (LAN) với nhau. \newline \textbullet \space Chuyển tiếp dữ liệu một cách có chủ đích giữa các thiết bị với nhau. \newline \textbullet \space Là thiết bị đóng vài trò tạo nên xương sống của hệ thống mạng LAN. \\
        \midrule
        Router              & \textbullet \space Thiết bị dùng để cung cấp Internet của nhà mạng, dùng để kết nối vào Firewall để các thiết bị có thể truy cập Internet.                                                                                                                                         \\
        \midrule
        Access Point        & \textbullet \space Phát sóng Wi-Fi, cho phép các thiết bị không dây kết nối với nhau qua mạng nội bộ. \newline \textbullet \space Mở rộng vòng phủ sóng của mạng.                                                                                                                  \\
        \bottomrule
    \end{tabular}
    \caption{Phần Mềm - Quản Lý}
    \label{tab:softwaremanagement}
\end{table}

\subsection{Phần mềm ứng dụng và dịch vụ}

\begin{table}[H]
    \centering
    \begin{tabular}{p{0.2\textwidth} | p{0.7\textwidth}}
        \toprule
        \textbf{Thành Phần} & \textbf{Mô Tả \& Mục Đích}                                                                                                                                                                                                                                                         \\
        \hline \hline
        Firewall            & \textbullet \space Bảo vệ mạng nội bộ khỏi các mối đe dọa từ bên ngoài Internet. \newline \textbullet \space Kiểm soát lưu lượng mạng vào/ra dựa trên các quy tắc ưu tiên QoS (Quality of Service). \newline \textbullet \space Ngăn chặn truy cập trái phép, phần mềm độc hại.    \\
        \midrule
        Switch              & \textbullet \space Kết nối các thiết bị trong cùng một mạng nội bộ (LAN) với nhau. \newline \textbullet \space Chuyển tiếp dữ liệu một cách có chủ đích giữa các thiết bị với nhau. \newline \textbullet \space Là thiết bị đóng vài trò tạo nên xương sống của hệ thống mạng LAN. \\
        \midrule
        Router              & \textbullet \space Thiết bị dùng để cung cấp Internet của nhà mạng, dùng để kết nối vào Firewall để các thiết bị có thể truy cập Internet.                                                                                                                                         \\
        \midrule
        Access Point        & \textbullet \space Phát sóng Wi-Fi, cho phép các thiết bị không dây kết nối với nhau qua mạng nội bộ. \newline \textbullet \space Mở rộng vòng phủ sóng của mạng.                                                                                                                  \\
        \bottomrule
    \end{tabular}
    \caption{Phần Mềm - Ứng Dụng và Dịch Vụ}
    \label{tab:softwareapplication}
\end{table}

\section{Tài nguyên mạng}

\section{Quản trị và bảo mật}

\chapter{Chapter 3}
\label{cha:chapter3}

\blindtext

\section{Section 3.1}
\label{sec:section3.1}

\blindtext

\subsection{Subsection 3.1.1}
\label{subsec:subsection3.1.1}

\blindtext

\subsubsection{Subsubsection 3.1.1.1}
\label{subsubsec:subsubsection3.1.1.1}

\blindtext

% vim: set ff=unix ft=tex fenc=utf-8 ts=4 sw=4 tw=79 :

\chapter{Ý tưởng và công cụ xây dựng ứng dụng} % (fold)
\label{cha:ideadandtools}

\blindtext

\blindtext

\section{Mục đích xây dựng ứng dụng} % (fold)
\label{sec:whybuildapp}

\blindtext

\begin{newitemize}
    \item[\textbullet] Lorem ipsum dolor sit amet, consectetur adipiscing elit.
    \item[\textbullet] Lorem ipsum dolor sit amet, consectetur adipiscing elit.
    \item[\textbullet] Lorem ipsum dolor sit amet, consectetur adipiscing elit.
    \item[\textbullet] Lorem ipsum dolor sit amet, consectetur adipiscing elit.
    \item[\textbullet] Lorem ipsum dolor sit amet, consectetur adipiscing elit.
\end{newitemize}

\section{Những yêu cầu chức năng cần đạt được} % (fold)
\label{sec:requirementfeaturesachieved}

Những chức năng cơ bản mà chương trình cần đạt được:

\begin{newitemize}
    \item[\textbullet] Lorem ipsum dolor sit amet, consectetur adipiscing elit.
    \item[\textbullet] Lorem ipsum dolor sit amet, consectetur adipiscing elit.
    \item[\textbullet] Lorem ipsum dolor sit amet, consectetur adipiscing elit.
    \item[\textbullet] Lorem ipsum dolor sit amet, consectetur adipiscing elit.
\end{newitemize}

\blindtext

\section{Cơ sở kỹ thuật} % (fold)
\label{sec:techresource}

\blindtext

\blindtext

\section{Phương pháp phát triển} % (fold)
\label{sec:devmethod}

\blindtext

\blindtext

\subsection{Tổng quan dự án}
\label{sub:projoverview}

\blindtext

\begin{newitemize}
    \item[\textbullet] Lorem ipsum dolor sit amet, consectetur adipiscing elit.
    \item[\textbullet] Lorem ipsum dolor sit amet, consectetur adipiscing elit.
    \item[\textbullet] Lorem ipsum dolor sit amet, consectetur adipiscing elit.
    \item[\textbullet] Lorem ipsum dolor sit amet, consectetur adipiscing elit.
    \item[\textbullet] Lorem ipsum dolor sit amet, consectetur adipiscing elit.
\end{newitemize}

\subsection{Lựa chọn phương pháp phát triển}

\blindtext

\section{Nền tảng và công cụ} % (fold)
\label{sec:platformandtools}

\blindtext

\subsection{Ngôn ngữ phát triển} % (fold)
\label{sub:devlang}

\blindtext

\subsection{Nền tảng hệ điều hành} % (fold)
\label{sub:platform}

\blindtext

\subsection{Quản lý phiên bản và mã nguồn} % (fold)
\label{sub:versioncontrol}

\blindtext

\subsection{Môi trường phát triển} % (fold)
\label{sub:devenv}

\blindtext

% vim: set ff=unix ft=tex fenc=utf-8 ts=4 sw=4 tw=79 :

\chapter{KẾT LUẬN}
\label{cha:roadmapoverview}

\blindtext

\section{Tổng kết kết quả nghiên cứu}
\label{sec:techproblemresolve}

\blindtext

\section{Đánh giá chung}
\label{sec:otherrequirements}

\blindtext

\section{Định hướng và giải pháp}

\blindtext

\section{Bài học kinh nghiệm}

\blindtext

% vim: set ff=unix ft=tex fenc=utf-8 ts=4 sw=4 tw=79 :

\chapter{Xây dựng chức năng} % (fold)
\label{cha:devbasicfeatures}

\blindtext

\section{Sample Tables}
\label{sec:samplatables}

Please refer to the table \nameref{tab:uploadmethod} for more information.

\begin{table}[htbp]
    \centering
    \begin{tabular}{@{} c | l @{}}
        \toprule
        \textbf{Phương thức} & \textbf{Mục đích} \\
        \midrule
        \textbf{POST} & Tải một ảnh hoặc \gls{url} \\
        \midrule
        \textbf{GET} & Tải một ảnh \\
        \bottomrule
    \end{tabular}
    \caption{Sample Table 1}
    \label{tab:uploadmethod}
\end{table}

\blindtext

Please refer to the table \nameref{tab:uploadparameter} for more information.

\begin{table}[htbp]
    \centering
    \begin{tabular}{@{}l | l | l | m{4cm}@{}}
        \toprule
        \textbf{Thông số} & \textbf{Bắt buộc} & \textbf{Phương thức} &
        \textbf{Ý nghĩa} \\
        \midrule
        \emph{key} & Bắt buộc & GET/POST & API key \\
        \emph{image} & Bắt buộc & POST & Image \\
        \midrule
        \emph{type} & Tùy chọn & POST & Image type \\
        \emph{name} & Tùy chọn & GET/POST & \\
        \emph{title} & Tùy chọn & GET/POST & \\
        \emph{caption} & Tùy chọn & GET/POST & \\
        \bottomrule
    \end{tabular}
    \caption{Sample Table 2}
    \label{tab:uploadparameter}
\end{table}

\subsection{Sample Codes}
\label{sub:samplecodes}

\blindtext

Please refer to the code in \nameref{lst:samplecode1} and \nameref{lst:samplecode2} for more information.

\subsubsection{Sample Code 1}
\label{sub:samplecode1}

\blindtext

\begin{listing}[!htbp]
\begin{minted}[frame=lines,fontsize=\footnotesize]{python}
import cStringIO
import pycurl

from xml.etree import cElementTree as ET
\end{minted}
\caption{Sample Code 1}
\label{lst:samplecode1}
\end{listing}

\subsubsection{Sample Code 2}
\label{sub:samplecode2}

\blindtext

\begin{listing}[htbp]
\begin{minted}[frame=lines,fontsize=\footnotesize]{python}
# Image file to upload
filename = "omgvoz.jpg"
\end{minted}
\caption{Sample Code 2}
\label{lst:samplecode2}
\end{listing}

%\chapter{Kết luận}
\label{cha:conclusion}

\blindtext

\section{Kết quả đạt được}

\blindtext

\section{Hạn chế}

\blindtext

\section{Hướng phát triển}

\blindtext

\section{Git repo}

Git repo của ứng dụng tại: \url{https://github.com/example/projectname}

% vim: set ff=unix ft=tex fenc=utf-8 ts=4 sw=4 tw=79 :

%\clearpage
%==============================================================================
% Now begin the Appendices, including them as separate files
\fancyhead{}
\pagestyle{fancy}
\renewcommand{\chaptermark}[1]{\markboth{#1}{}}
\renewcommand{\sectionmark}[1]{\markright{#1}{}}
\lhead{\textsc{\chaptername} \space \thechapter. \space \textsc{\leftmark}}
\rhead{\thepage}
\renewcommand{\footrulewidth}{0.4pt} % default is 0pt
\lfoot{\small GVHD:~\instructorname}
\rfoot{\small SVTH:~\studentname}

%\appsecnums
\appendix % Cue to tell LaTeX that the following 'chapters' are Appendices
% \begin{appendices}
%% Appendix A
\chapter{Tiêu đề của Phụ lục A}
\label{cha:appendixA}

\section{Tiêu đề của section A.1}

% TODO: No image/figure border.

\begin{figure}[htb]
	\centering
	\includegraphics[width=\textwidth,height=\textheight,keepaspectratio]{./graphics/tiobeindex.png}
	\caption{Sample Figure 1}
	\label{fig:appendixasamplefigure1}
\end{figure}

\pagebreak

\section{Tiêu đề của section A.2}

\begin{listing}[hbtp]
    \inputminted[frame=lines,fontsize=\footnotesize]{xml}{./data/datafile.xml}
    \caption{Sample Code 1}
    \label{lst:appendixasamplecode1}
\end{listing}
	% Appendix Title
%\chapter{Tiêu đề của Phụ lục B}
\label{cha:appendixB}

\section{Tiêu đề của section B.1}

\blindtext

\subsection{Tiêu đề của subsection B.1.1}

\blindtext

\subsection{Tiêu đề của subsection B.1.2}

\blindtext

\subsection{Tiêu đề của subsection B.1.3}

\blindtext
	% Appendix Title
% \begin{appendices}
%\backmatter
\addtocontents{toc}{} % Add a gap in the Contents, for aesthetics
\clearpage
%==============================================================================
\pagestyle{fancy}
\fancyhead{}
\addtotoc{Tài Liệu Tham Khảo}
%\label{bibliography}
% Change the left side page header to "Bibliography"
\lhead{\emph{Tài Liệu Tham Khảo}}
% Use the "unsrtnat" BibTeX style for formatting the Bibliography
\bibliographystyle{unsrtnat}
% The references (bibliography) information are
\bibliography{./bibliography/bibliography}
%==============================================================================
\end{document}  % THE END
%==============================================================================
